\chapter{Cardiac biology}
\label{chapter:intro-heart}
In \cref{chapter:GWAS-3Dheart} and \cref{chapter:GWAS-FD}, I investigate genetic associations of human heart morpholgy. To aid with an understanding of the relevant biology and key terms, I use this chapter to give a basic overview of human heart morphology, cardiovascular diseases and their underlying genetics.

The human heart is composed of four chambers, the left and right ventricle and the left and right atrium. On the outside it is covered by a tough membraneous structure, the pericardium. The innermost layer of the  pericardium, the epicardium, is fused to the heart and forms part of the heart wall. It directly connects to the myocardium, the thickest layer of the heart wall which is composed of conductory and contractile cardiomyocytes. On the inside, the myocardium is lined by the endocardium \citep{Betts2013}.  

The four chambers of the heart (\cref{fig:heart}) are separated through two septal structures, the interventricular and the aterioventricular septum. The blood exchange between the atria and ventricles is enabled through a set of valves embedded in the aterioventricular septum: the mitral valve between left atrium, and ventricle and the tricupsid valve between the right atrium and ventricle. In addition, each ventricle has a valve at its exit point. In the right ventricle, the pulmonary valve separates the ventricle from the pulmonary artery. Similarily, the aortic valve separates the left ventricle from the aorta. There is no direct blood exchange between the left and the right side of the heart in healthy individuals. 

Diseases of the heart are common and one of the leading health issues world-wide. They include a wide range of disorders from atherosclerosis, diseases of the myocardium and the heart’s electrical circuit to congenital heart diseases. To help with an understanding of these disease pathologies, the circulatory and conductary system as well as the development of the heart are described below.

\begin{figure}[htbp]
	\centering
	\includegraphics[trim = 0mm 0mm 0mm 0mm, clip, width=\textwidth]{Introduction/Figures/heart.pdf}
	\caption[\textbf{Anatomy, circulatory and conductory system of the human heart. }]{\textbf{Anatomy, circulatory and conductory system of the human heart. }A. Circulatory system. Deoxygenated blood (blue arrows) arrives at the right ventricle (RV) from the systemic circulation. From the right atrium, it enters the right ventricle (RV) through the tricupsid valve. It leaves the RV through the pulmonary valve into the pulmonary artery entering the pulmonary circuit. Oxygenated in the lung, blood (red arrows) arrives back at the heart at the left atrium (LA) through two branches of the vena cava and enters the left ventricle (LV) passing the mitral valve. It leaves the left ventricle through the aorta, entering the systemic circulation. Anotomy: The myocardium of the left ventricle is significantly thicker than the right ventricle, as it has to overcome greater pressure of the systemic circuit. The walls of the atria are smooth, whereas the ventricles show protrusions. The atrioventricular septum separating atria and ventricles is not shown for simplicity. It is located at the level of the tricupsid and mitral valves. B. Conductory system. The sinoatrial (SA) node initiates the contraction of the heart by sending an action potential through the atria via cell-cell contact and specific pathways (Bachmann's Bundle and internodal paths). The potential arrives at the atrioventricular node, where it is delayed to allow for full contraction of the atria before it is passed on to the Purkinje Fibers, through the Bundle of His and the atriventricular bundle branches. The Purkinje Fibers pass the signal on to the ventricles, leading to their contraction and the pumping of the blood outside of the heart. } 
	 	\label{fig:heart}
\end{figure}
%
 
%\section{Circulation}
%In the healthy human heart, the blood flow is unidrectional.  Blood coming from the systemic circulation (to and from the body) enters the heart through the superior and inferior vena cava and is collected in the right ventricle. Upon contraction of the atria, the blood is pushed into the right ventricle, from which it leaves the heart through the pulmonary artery, entering the pulmonary circulation (to and from the lung). Blood coming from the body and leaving to the lung is relatively deoxygenated and enriched for carbon dioxide. In the lung, carbon dioxide is exchanged for oxygen and the blood returns to the left side of the heart via pulmonary veins. It enters the heart through the left atrium and is subsequently pushed into the left ventricle. From there, the oxygenated blood is pumped into the systemic circulation, leaving the left ventricle through the aorta \citep{Betts2013}. 

\section{Cardiac cycle}
The cardiac cycle begins with the contraction of the atria and ends with the relaxation of the ventricles. During the cycle, the chambers of the heart can be found in two distinct states, systole and diastole. In systole, the chambers contract and pump blood into either the ventricles (atria) or out of the heart (ventricle). In diastole, the chambers are relaxed and fill with blood. Both atria and ventricle cycle through these states, coordinated by impulses sent from the circulatory system.  Ventricles are in diastole when atria undergo systole and vice versa. In atrial diastole, the valves separating atria and ventricles are open and facilitate passive blood flow into the ventricles. When the cardiac cycle starts, atria enter systole and pump the remaining blood into ventricles.  The amount of blood contained in the ventricles at the end of atrial systole/ventricular diastole is refered to as end diastolic volume (EDV). When the ventricle enter systole, the pressure in the ventricles rise compared to the one in the atria which are in diastole and the separating valves are closed as a response to the increased pressure. Once the ventricular pressure overcomes the pressure in aorta and pulmonary arteries, the respective valves open and equivalent amounts of blood are pumped into the systemic and pulmonary cycle. The larger and higher resistance vessels of the systemic circulation compared to the low-pressure vessels of the pulmonary system put a higher demand on the left ventricle which is met by a proportionally higher mass of the left ventricle compared to the right. The amount of blood that each ventricle can pump within one cardiac cycle is defined as the stroke volume (ST). The volume of blood remaining in the ventricle at the end of systole is referred to as end systolic volume (EST). Together with DST and ST, EST is an important clincal parameter \citep{Betts2013}. 

\section{Conduction system}
The conduction system of the heart establishes the heart rythm through electrical impulses sent by specialised myocardial conducting cells. The normal cardiac rhythm, called sinus rhythm, is established by the sinoatrial node and is located at the junction of the superior vena cava and the right atrium (\cref{fig:heart}\subfig{B}). The sinus node is also called the pacemaker of the heart, since the signal leading to the activation of the myocardial contractile cells and, in consequence, their contraction starts here. Upon initiation of the action potential in the sinus node, the depolarisation spreads through the atria to the atrioventricular node via cell-cell contacts, the internodel pathways and Bachmann's bundle \citep{Laske2005,Anderson2009}. The atrioventricular node is located within the atrioventricular septum which prevents the signal to spread directly to the ventricle without being processed. At the atrioventricular node, the signal is delayed to allow the atria to complete their contraction which pumps the blood into the ventricles. From the atrioventricular node, the signal is propagated along the interventricular septum through the bundle of His which divides into the atriventricular bundle branches. These in turn connect with Purkinje Fibers at the apex of heart, which propagate the impulse to the myocardial contractile cells in the ventricles. The contraction of the ventricles follows the direction of the impulse and travels from the apex towards the base, pumping blood out of the ventricles and into the aorta and pulmonary arteries \citep{Laske2005,Sigg2010}.  

\section{Heart development}
\label{section:heart-development}
The heart is the first functional embyronic organ and already starts to beat by the end of the third week of development \citep{Zambrano2002}. In the developing heart, three major processes have to be orchestrated: the formation and arrangement of the myocardium into the four-chamber heart, the development of the conduction system, and the heart's circulatory system required for nutrition and oxygen supply to the myocardium. The first two processes happen simultaneously, while the latter can only take place after proper development of the myocardium. 

The development of the heart starts in the third week of development, just after gastrulation. In gastrulation, the single-layered sheet of epithelial cells that forms the embryo is re-organised into three germ layers, the ectoderm (external layer), mesoderm  (middle layer) and endoderm (internal layer). Each layer will give rise to different tissues and organs in the developing embryo. The heart development begins with the formation of two cardiac crescents from the mesodermal layer (\cref{fig:dev-heart}, \subfig{1}), which are located near the head of the embryo \citep{Christoffels2000}. Within each cardiac crescent, two structures develop, a plate of myocardial cells and a plexus of endothelial strands. These develop into cardiogenic cords, with the endothelial strands forming a tube structure enveloped  by a layer of myocardial cells. By the fusion of the two cardiogenic cords, the early tubular heart is formed (\cref{fig:dev-heart}, \subfig{2}). This early tubular structure already shows peristaltic contraction, despite the lack of valves and conduction system \citep{Goss1938,deJong1992,Moorman1994}. The tubular heart then undergoes a right-ward looping where an initial differentiation into ventricular myocard, atrial myocard and transitional zones occurs (\cref{fig:dev-heart}, \subfig{3}). The transitional zones will form parts of the septa, valves, conduction system and fibrous heart skeleton \citep{Gittenberger-deGroot2005}. Through the looping of the heart an inner and an outer curvature is created. The developing atria and ventricle stand out on the outer curvature, whereas transitional zones are brought into proximity on the inner curvature (\cref{fig:dev-heart}, \subfig{4}). 

\begin{figure}[hbtp]
	\centering
	\includegraphics[trim = 0mm 0mm 0mm 0mm, clip, width=\textwidth]{Introduction/Figures/heart-dev.pdf}
	\caption[\textbf{Embryonic heart development. }]{\textbf{Embryonic heart development. } 1. The mesoderm gives rise to two cardiac crescents that already show some extend of asymmetry. 2. The cardiac crescent have fused together to from a straight heart tube. 3. The straight heart tube starts a right-ward looping. Parts marked in red will develop into the ventricles, while parts maked in turquoise will become atria. 4. The looping heart with precursors of the atria (A), the left ventricle (LV), the right ventricle (RV) and the outflow tract(OFT). Ring-like structures mark the transitional zones: sinoatrial ring (SAR), atrioventricular ring (AVR), primary ring (PR), ventriculararterial ring (VAR).} 
	 	\label{fig:dev-heart}
\end{figure}

Correct looping and positioning ot the transitional zones are critical for the separation of the heart into its functional components. The separation is facillated through septation at the atria, the ventricles and the arterial pole. For the separation of the ventricles, two processes have to be considered, the inflow and outflow septation. The inflow septation i.e. the septation of the ventricles from one another and from the atria, is mainly achieved through the primary ring. The primary ring gives rise to the the ventricular septum that separates left and right ventricle. This process has to be orchastrated with the position of atrioventricular ring, which is pulled towards the right ventricle by a tightening of the inner curvature. The positioning of the atrioventricular ring above the left and right ventricle builds the base for the formation of the mitral and tricupsid valve, respectively, which will separate the atria from the ventricles.  The septation controlling the blood flow from ventricles to the arteries (outflow septation) is achieved through the twisting of the ventricularaterial ring into the precursors of the pulmonic and aortic valve and their positioning above the right and left ventricle.  

At the end of week nine in development, the heart consists of the four chambers divided by septa with integrated valves. Morphologically, atria and ventricle can be distinguished based on the structure of their myocard. While the myocardium of the atria is thin and has a smooth surface, the ventricles show a much thicker myocardium with portrusion (trabeculations) running along the endocardial surface. 

During these rearrangement processes the myocardium also underwent a differentiation into the contracting and conducting myocardium. While many components of the gene regulatory networks that control the differentiation are known today,  mechanisms involved in controlling this differentiation on a cellular and region-specific level remain to be discovered \citep{Christoffels2009,Paige2015,Park2017}. Structures important in the development of the conduction system are the sinoatrial ring which will develop into the sinoatrial node, the primary ring which will give rise to the atrioventricular conduction system and the atrioventricular ring developing into Bachmann's Bundles. 

\section{Common cardiovascular diseases}
\label{subsection:CVD}
According to the International Statistical Classification of Diseases and Related Health Problems (ICD), the cassification system of the world health organisation (WHO), total cardiovascular diseases include hypertension, hypercholesterolemia, coronary heart disease, cardiac arrhythmias, congenital heart diseases and cardiomyopathies (ICD-10 codes I00-I99, Q20-28 \citep{WHO2016}). 

The largest contribution to cardiovascular diseases are coronary heart diseases. Their major clinical manifestations are myocardical infarction (commonly known as heart attack), angina pectoris (chest pain), and sudden coronary death \citep{Wong2014}. The common cause of coronary heart diseases is an interrupted blood and consequently oxygen supply to the heart through a blockage of the coronary arteries. Major risk factors are high blood pressue (hypertension) and high blood cholesterol (hypercholesterolemia) \citep{Mackay2004}.

Cardiac arrhythmias are a class of diseases where the observed cardiac rhythm is different from the regular sinus rhythm. They are caused by irregularities of impulse generation and/or conduction. Tachycardia is the condition of an increased heart rate whereas bradicardia describes a lower than normal heart rate. They can cause a reduction in cardiac output and myocardial blood flow and may be life-threatening \citep{Durham2002}.
%In adults these are defined as three or more consecutive impulses at a rate higher than 100bpm or lower than 60bpm repectively.  

Congenital heart diseases are diseases with structural abnormalities of the heart or intrathoraic great vessels that are of functional significance and have been present since birth \citep{Mitchell1971}. They may be caused by genetic or environmental factors during pregnancy and include ventricular outflow tract  obstructions i.e. narrow or blocked arteries and valves and septal defects. Of the latter, interventricular septal defects are the most common \citep{Hoffman2005}.

%Other forms of heart diseases covers a wide range, from inflammatory heart disease and heart failure to myopathies. Inflammatory heart diseases can affect each layer of the herat wall individually (pericarditis, myocarditis and endocarditis) and while they might be caused by metabolic or immune diseases, most commonly they are induced by external factos such bacterial or viral infection and toxins \citep{Lu2015}. 

Cardiomyopathies describe a class of diseases where the heart muscle fails to function properly. Traditionally, they are classified based on their anatomy and hemodynamics into hypertrophic, dilated, or restrictive cardiomyopathy. The incidence of the latter is rare and no changes in ventricular morphology are observed. This is in stark contrast to hypertrophic and dilated forms, where an increase in ventricular wall thickness or volume are observed, respectively. The increase in wall thickness is caused by a hypertrophy of existing myocytes rather than a hyperplasy as in the developing heart \citep{Lorell2000}. Dilated cardiomyopathy presents with an increase in cardiac chamber volume and often a modest increase in wall thickness. Both mechanism are in response to cardiac stress and initially improve heart function but in the long run increase myocardial strain and raise metabolic demands \citep{Seidman2001}.

Cardiovascalur diseases are caused by a combination of environmental and genetic risk factors. Amongst the environmental risk factors one can distinguish between modifiable risks governed by the individual itself and exposure to risk factors which are often beyond the influence of the indivdual. The latter include exposure to solvents, pesticides or extremes in noise and temperature \citep{Bhatnagar2004,Brook2010,Babisch2014}. Modifiable risk behavior such as smoking, physical inactivity and a poor diet have been shown to be highly correlated with the incidence of cardiovascular diseases (reviewed in \citep{OToole2008,Cosselman2015}). Meta-studies examining behavioral change in the English,Welsh and American populations over a period of 20 years, have shown a decline in coronary heart disease mortality due to a reduction in smoking, increased physical activity and other behavioral factors \citep{Unal2004,Ford2007}. Genetic risk factors for cardivascular diseases are described in the next section. 

%The type of hypertrophy in the myocytes depends on the nature of th hemodynamic burden. An increase in pressure on the myocard caused by for instance hypertension leads to the parallel addition of sarcomers. This additiona causes an increase in myocyte width and consequently an increase in wall thickness. Hemodynamic burden on the heart due to volume increase gives rise to myocyte lengthening by sarcomere replication, leading to an increase in ventricular volume

\section{Genetics of cardiovascular diseases}
The genetics of cardiovascular diseases point both to simple Mendelian and complex inheritance patterns. In multiple linkage analyses studies of familial myocardial hypertrophy, several genes have been discovered where mutations segregate in a Mendelian fashion. These include mutations in cardiac myosin heavy chain (MHC) \citep{Geisterfer-Lowrance1990}, \(\alpha\) tropomyosin, cardiac troponin T and C, \citep{Thierfelder1994, Kimura1997} and cardiac mysosin binding protein \citep{Carrier1993,Bonne1995}. Another group of familial cardiovascular diseases, familial hypertension, has been linked to muations in epithelial sodium channels  SCNN-2 and SCNN3-3 \citep{Boyden2012,Glover2014} as well as KLH3-CUL3, genes coding for proteins building a complex involved in Sodium-chloride reabsorbtion in the kidney  \citep{Hansson1995}. Linkage studies have also pinpointed genes for atrial and ventricular septal defects. They are linked to mutations in the transcription factors,  GATA4 \citep{Schott1998} and NKX2-5 \citep{Garg2003}, respectively. 

In contrast, the majority of cardiovascular traits follow complex inheritance pattern with interaction between multiple genes and non-genetic factors \citep{Kathiresan2012}. Genome-wide association studies have been successful in finding genetic loci associated with a large number of cardiovascular diseases. Out of the \num{4148} studies in the GWAS catalogue (accessed 11.08.2017), \num{159} contain phenotype descriptions relating to cardiovascular diseases (list of query terms in \cref{tab:gwas-studies} in the appendix).
\\
\begin{figure}[hbtp]
	\centering
	\includegraphics[trim = 0mm 0mm 0mm 0mm, clip, width=0.8\textwidth]{Introduction/Figures/GWASheartstudies.pdf}
	\caption[\textbf{GWAS on heart-related phenotypes. }]{\textbf{GWAS on heart-related phenotypes. } Overview of \num{153} GWAS studies with \num{59} unique heart-related phenotypes (obtained from the GWAS catalogue \citep[accessed on 11.08.2017]{MacArthur2017}). Phenotypes were grouped into eight phenotype classes. The list of query terms and their grouping can be found in \cref{tab:gwas-studies} in the appendix.} 
	 	\label{fig:gwas-heart}
\end{figure}

The highest number of studies has been conducted on blood pressure phenotypes, followed by electrocardiographic traits and coronary heart diseases (\cref{fig:gwas-heart}). Early GWAS on these traits were conducted on samples of the Framingham heart study, a community-based cohort study founded in 1948 to examine the epidemiology of cardiovascular disease \citep{Dawber1951,Kannel1979}. The Framingham Heart Study 100K SNP genome-wide association study resource was published in 2007 \citep{Cupples2007} and its \num{1345} participants built the basis for \num{17} genome-wide assocation studies on traits like echochardiographic dimension \citep{Vasan2007}, blood pressure \citep{Levy2007} and heart rate \citep{Newton-Cheh2007}. Later studies often contained larger sample sizes or re-analysed previously published studies in meta-analysis. For instance, the international consortium for blood pressure conducted a meta-anlaysis of \num{29} previously published GWAS on systolic and diastolic blood pressure phenotypes and discovered \num{16} novel loci, ten of which were associated with known blood pressure-related genes \citep{Ehret2011}. Similarily, the large consortium for coronary heart diseases  (CARDIoGRAM)  conducted a case-control meta-analysis  and identified ten novel loci  \citep{Nikpay2015}. The other classes of phenotypes are smaller and more heterogeneous, comprising different congenital heart diseases e.g. congenital left-sided heart lesion \citep{Mitchell2015,Hanchard2016} and conotruncal heart defects (i.e. malformations of the cardiac outflow tracts) \citep{Agopian2014} or morphological traits including cardiomyopathies \citep{ Villard2011} and cardiac wall thickness \citep{Vasan2009,Arnett2011}. 

\section{Thesis outline}
In the following chapters, I describe new methods and applications for the genetic analysis of high-dimensional datasets. 

In \cref{chapter:simulation}, I introduce the R package that I developed for the simulation of complex phenotype structures. Simulated phenotypes serve as an approximation for observed biological phenotypes and are invaluable for model developement. All phenotypes simulated in this thesis are generated based on the strategies described in this chapter. The simulation strategy and applications have been summarised in a publication \citep[\textit{under revision}]{Meyer2017a}.

\Cref{chapter:limmbo} presents LiMMBo, a new approach for finding genetic associations in high-dimensional phenotypes using linear mixed models. I first demonstrate model calibration and power on simulated datasets before I apply LiMMBo to a publically available dataset of yeast growth traits in \cref{chapter:yeast}. A manuscript of LiMMBo and its application has been in preparation at the time of thesis submission.

In \cref{chapter:DimReduction}, I systematically analysed twelve unsupervised dimensionality reduction methods for their ability to find robust phenotype representations of simulated data with different structure and size. I introduce a new stability measure for choosing the low-dimensional representations and demonstrate that the selected representation can recover genetic associations.

Finally, I investigate genetic associations for human heart morphology based on MRI data of \num{1500} healthy individuals. In \cref{chapter:GWAS-3Dheart}, I apply the methods and measures described in  \cref{chapter:DimReduction} to obtain a low-dimensional representation of the heart morphology and conduct a genome-wide association study based on this representation. \Cref{chapter:GWAS-FD} describes the genome-wide association study on a cardiac trabeculation phenotype derived from a supervised feature extraction approach on the MRI data.  The work in these chapters was done in collaboration with Antonio De Marvao, Jiashen Cai, Pawel Tokarczuk Declan O'Regan and Stuart Cook from Imperial College London. Specifically, phenotype acquisition and feature extraction was done by my collaborators, while I was responsible for all remaining analyses, including genotype qualtiy control and imputation. An initial paper using the imputed genotypes was recently published \citep{Biffi2017}.




%, suggesting additive and potentially interaction between genetic effects. Environmental risk factors and gene-environment interactions introduce another level of complexity for finding causal genes and loci. 

