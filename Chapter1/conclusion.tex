\subsection{Conclusion}
The simulation results show that Limmbo is effective in estimating large covariance matrices for multi-trait analysis. In a real, published dataset from yeast, I can show an increase in power by detecting new loci that could not be associated in single-trait analysis. As well as the gain in power, the analysis provides a more integrated view of the relationships between loci, with a number of loci showing similar beta-patterns, suggesting that they are part of the same biological mechanism. Interestingly, all linear mixed models probably require some level of population structure to form good estimates of the trait-by-trait covariance structure of the background effects; in situations with low levels of population structure the simpler approach of a linear model without a covariance matrix is better calibrated. Overall I have shown that Limmbo is a robust method for large scale trait analysis with benefits in real world settings.