\newpage
\section{Multi-trait GWAS with LiMMBo}
In order to show the utility of LiMMBo for joint high-dimensional phenotype analyses and to demonstrate the advantages over single-trait approaches, I analysed the imputed dataset both with single-trait and multi-trait GWAS. 

\subsection{LiMMBo increases power in detecting genetic associations}
For both analyses, I used a LMM where the sample-by-sample component of the random genetic effect is based on the genetic relationship matrix. To obtain an estimate of the genetic relationship matrix, I first pruned the genome-wide SNPs (\num{11623}) for SNPs that are in LD within a window of 3kb and show a correlation \(r^2 > 0.2\) . As the dataset is based on an F2 cross, LD structure estimation is not straight-forward and this window size is a simple estimate derived from a study on the population genomics of domestic and wild yeasts \citep{Liti2009}. The LD pruning reduced the SNP set for GRM estimation to \num{4105} SNPs. The GRM was estimated by the method introcuded by Yang and colleagues \citep{Yang2011} (\red{~\cref{section:grm}}). Both GRM estimation and LD pruning was done via \textit{plink} \cite{Chang2015}.

The first step in the mtGWAS is the trait-by-trait covariance estimation via LiMMBo. \num{1000} bootstraps of \num{10} traits each were run and their trait-by-trait covariance estimated. The combined trait-trait covariance estimates \tmatsub{C}{g} and \tmatsub{C}{n} were used as input estimates for the second step in the mtGWAS, the mvLMM  (\cref{eq:mvLMM}) across all genome-wide SNPs. \red{Runtime for covariance estimation.} I used a mvLMM with a trait-design matrix corresponding to the any effect test, i.e. testing for an effect of the SNP on any of the traits compared versus the null hypothesis of no association (\red{\cref{sec:mvtests}}).

For the stGWAS, the trait-by-trait components of the random effects are point estimates (\(\sigma_g\) and \(\sigma_n\)) derived within the LMM and do not require \textit{a priori} estimation. The stGWAS is based on a univariate LMM (\cref{eq:uvLMM}) per SNP, where each trait is mapped individually. To account for the number of univariate tests, the p-values obtained from the stGWAS were adjusted for multiple testing by the effective number of conducted tests \(M_\text{eff}\). \(M_\text{eff}\) was introduced by Galwey and colleagues \citeyear{Galwey2009} and adjusts for multiple testing in a manner similar to the Bonferroni method \citep{Dunn1961}. However, it is less conservative, as it does not adjust for total number of tests, but the estimated, effective number of tests, taking correlation between the 
variables and tests into account:
\begin{equation}
 M_\text{eff} = \frac{(\sum^M_{i=1} \sqrt{\lambda_i})^2}{\sum^M_{i=1}\lambda_i},
 \label{eq:meff}
\end{equation}
 where \(\lambda\) are the eigenvalues of the phenotypes' correlation matrix. To adjust for multiple testing in the stGWAS, the p-values are multiplied with \(M_\text{eff}\) and set to \num{1} if the multiplication leads to values greater than \num{1}. \(M_\text{eff}\) for the \num{41} growth traits was estimated to be \num{33}. 
 
In order to assess the significance of the single-trait and multi-trait analyses, I followed approaches of previous association studies in yeast crosses \citep{Brem2002,Brem2005,Ehrenreich2010}, where permutations are used to estimate empirical significance levels. With a conservative, theoretical singificance threshold of \(p_\text{t}=10^{-5}\), at most one SNP is expected to be false positive in a total of \(s = 11,623\) SNPs. To find the empircal FDR corresponding to this threshold, I generated \(k = 50\) permutations of the genotypes and fitted the LMMs against these permutations. The p-values obtained from these analyses were then combined and sorted in increasing order. The p-value observed at position \(p_\text{t} \times s \times k\) is used as the empirical FDR. For the mtGWAS this threshold is at \(\text{FDR}_{\text{mtGWAS}} =1.2 \times 10^{-5}\)  lower than for the stGWAS \(\text{FDR}_{\text{stGWAS}} =8.6 \times 10^{-6}\). \red{I am quite certain that this is the right way to do, but when I wrote this I got really confused that the empirical FDR is lower then the theoretical threshold, need to discuss}.

\Cref{fig:GWAS-yeast} shows the manhattan plot of the multi-trait and single-trait GWAS. On several chromsomomes (e.g. \numlist{1;6;15}, mtGWAS peaks (blue) are observed whereas no stGWAS peaks (orange; minimum p-value per SNP across all \num{41} stGWAS, adjusted for multiple testing) can be detected. On the other hand, there are a few significant loci for the stGWAS where the multi-trait analyses either does not reach significance (e.g. on chromsome \num{7}) or does not detect any association (e.g. on chromosome \num{4}). For these loci, the underlying genetics seem to be trait specific to magnesium sulfate and hydroquinone, respectively. Testing with a \num{41} degrees of freedom test as done in the mtGWAS hinders to detect these strong mono-trait associations. Both, the gain in power for the multi-trait associations and the burden of a multi-variate test when the underlying effect is uni-variate confirm the results obtained from the theoretical power analysis (\cref{sec:power}). 

\begin{figure}[hbtp]
	\centering
	\includegraphics[trim = 0mm 0mm 0mm 0mm, clip, width=\textwidth]{Chapter3/Figures/manhattan.png}
	\caption[\textbf{Manhattan plot of p-values from single-trait and multi-trait GWAS.}]{\textbf{Manhattan plot of p-values from single-trait and multi-trait GWAS.} The stGWAS p-values were adjusted for multiple testing by the effective number of tests (\(M_\text{eff} = 33\)) and only the minimum adjusted p-values across all \num{41} traits per SNP are shown. The significance line is drawn at the empirical \(\text{FDR}_{\text{stGWAS}} =8.6 \times 10^{-6}\).}
 	\label{fig:GWAS-yeast}
\end{figure}

To quantify the increase in power, I counted the number of loci detected above the permutation-based significant thresholds for both the stGWAS and the mtGWAS. However, for a fair comparison of the \num{2} methods, I also needed to account for linked loci, with the long LD structure present in the F2 cross potentially merging nearby signals. As for the estimation of the GRM, I used \textit{plink} for LD pruning of the SNPs, chosing a strict threshold of \(r^2 > 0.2\) and increasing LD window sizes ranging from \numrange{3}{100}kb.  The maximal LD window of \num{100}kb covers between \num{6}\% (chromosome \num{4}) and \num{43}\% (chromosome 1) of total chromosome length (yeast genome assembly: ScerevisaeR64-1-1 \red{\citep{}}). \Cref{tab:sigsnps} shows that the increase in power is present from narrow to broad LD pruning, with on average \num{29}\% more significant loci in mtGWAS.

% Table generated by Excel2LaTeX from sheet 'LDGwasYeast'
\begin{table}[htbp]
  \centering
  \caption[\textbf{Comparison of significant loci in single-trait and multi-trait GWAS.}]{\textbf{Comparison of significant loci in single-trait and multi-trait GWAS.} In the column `All SNPs', the absolute number of SNPs beyond the significance threshold for multi-tait and single-trait GWAS as well as their ratio (multi-trait:single-trait) are depicted. In order to limit the potential bias in the counting of the loci, introduced by different degrees of linkage disequilibrium (LD) for different loci, the genome-wide SNPs were LD pruned and the ratio of significant SNPs determined for five different LD window sizes. The maximal LD window covers between \num{6}\% (chromosome \num{4}) and \num{43}\% (chromosome \num{1}) of total chromosome length.}
\begin{tabular}{lllllll}
    \toprule
          & \multicolumn{1}{c}{\multirow{2}[4]{*}{All SNPs}} & \multicolumn{5}{c}{LD pruned with $r^2 \ge 0.2$} \\
\cmidrule{3-7}          &       & \multicolumn{1}{r}{\num{3}kb} & \multicolumn{1}{r}{\num{10}kb} & \multicolumn{1}{r}{\num{30}kb} & \multicolumn{1}{r}{\num{50}kb} & \multicolumn{1}{r}{\num{100}kb} \\
    \midrule
    NrSNPs & \num{11623} & \num{4105} & \num{1028} & \num{264} & \num{161} & \num{107} \\
    multitrait & \num{1132} & \num{384} & \num{101} & \num{24} & \num{15} & \num{9} \\
    singletrait & \num{695} & \num{275} & \num{72} & \num{20} & \num{13} & \num{7} \\
    multitrait:singletrait & \num{1.63} & \num{1.4} & \num{1.4} & \num{1.2} & \num{1.15} & \num{1.29} \\
    \bottomrule
    \end{tabular}%
  \label{tab:sigsnps}%
\end{table}%

\subsection{Multi-trait effect size estimates as indicators for common biology}
As well as providing an increase in power, the mtGWAS inherently provides effect size estimates across all phenotypes for a particular locus, allowing for a richer exploration of pleitropic effects of each of loci.
To analyse the relationship between traits and SNPs based on their effect size estimates, I filtered the genome-wide SNPs for SNPs that fell within a gene body (yeast genome assembly: ScerevisaeR64-1-1) and pruned these \num{8135} SNPs for SNPs in LD  with \(r^2 > 0.2\) and within a \num{3}kb window (\num{1412} SNPs). Lastly, I filtered for SNPs passing the \(\text{FDR}_\text{mtGWAS} =1.2 \times 10^{-5}\) yielding \num{210} SNPs across \num{15} out of the \num{16} yeast chromosomes. Chromsome \num{5} is the only chromosome without significantly associated SNPs for single-trait and multi-trait GWAS (\cref{fig:GWAS-yeast}). 

To find groups of SNPs and traits with similar effect size estimates, I clustered the effect size estimates of these SNPs both across traits and SNPs. I used the hierarchical clustering algorithm \textit{pvclust} that provides bootstrap-based p-values as a measure for the stability of a given cluster \cite{Suzuki2006}. The clustering was based on their Pearson correlation coefficients, with \num{50000} bootstraps for traits and \num{10000} for SNPs. Clusters with \(p < 0.05\) were considered significant. A heatmap of effect size estimates and the clustering results is depicted in \cref{fig:effectsizes}. Ignoring the clustering for a first impression of the results, one can clearly see that most SNPs have significant, non-zero effects in more than one trait. Furthermore some traits have contributions from across the genome, many of which are xenobiotic growth conditions e.g. zeocin \cite{Krol2015} and neomycin \cite{Alamgir2010}. Turning the attention to the clustering, \cref{fig:effectsizes} shows that the clusters are driven by specific combinations of loci and traits, and would be hard to achieve from a single-trait analysis. 

There are a number of significant clusters for traits (\cref{fig:effectsizes}, red branches in the row dendrogram), including classically linked carbon metabolism sources (lactose, lactate and ethanol), and other clusters which there is literature support for. For example, I found a study showing gene expression changes for genes involved in DNA replication upon treatment with hydroxyurea and \num{4}-nitroquinoline-l-oxide (x4NQO) \citep{Elledge1990}, \num{2} substances that form a stable cluster. A study demonstrating trehalose and sorbitol to have synergistic effects on viability in yeast \citep{Hua2015} supports the cluster formed by these sugars. For other clusters, such as SDS and Hydroxybenzaldehyde or magnesium sulfate and berbamine I was unable to find literature support but this could be a candidate clustering of these growth phenotypes for further investigation. 

I discovered \num{31} stable SNP clusters (\cref{fig:effectsizes}, red branches in the column dendrogram), many of which represent linked loci. However, there are nine clusters (\cref{fig:effectsizes}, grey boxes) spanning multiple chromosomes, and many clusters with disjoint regions across a chromosome. Some SNP clusters have suggestive common annotation, such as \textit{cluster a} which has \num{2} members of the nuclear pore complex, and \textit{cluster b} which has a common set of vesicle associated genes (\cref{fig:effectsizes}, labeled boxes). The small size of clusters inhibited any systematic gene ontology based enrichment, but the ability to explore both multiple traits and multiple loci from the mtGWAS provides stimulating hypothesis generation. 

\begin{figure}[hbtp]
	\centering
	\includegraphics[trim = 0mm 150mm 0mm 0mm, clip, width=\textwidth]{Chapter3/Figures/effectsizes.png}
	\caption[\textbf{Hierarchical clustering of mtGWAS effects size estimates.}]{\textbf{Hierarchical clustering of mtGWAS effects size estimates.} Effect size estimates of LD-pruned (\num{3}kb window, \(r^2 > 0.2\)), significant SNPs located within a gene body were clustered by loci and traits (both hierarchical, average-linkage clustering of Pearson correlation coefficients ). Stable clusters (pvclust \( p < 0.05\)) are marked in orange. Grey boxes indicate stable SNP clusters spread across at least \num{2} chromosomes. a and b label \num{2} clusters for which suggestive common annotation was found, for details see text.}
 	\label{fig:effectsizes}
\end{figure}

\section{Summary}
A particular benefit of LMMs in general is that complex genetic relationships can be modelled, which is useful in structured populations such as this \(F_2\) cross in yeast. In uni-variate LMMs, the kinship information is used to account for background genetic effects in associations with a single trait. When used in multi-variate LMMs, the kinship structure allows for the estimation of complex trait covariance structure. However, only through a method like LiMMBo is it possible to efficiently map all \num{41} growth traits together and investigate pleiotropic effects on a genome-wide level. I demonstrated that such a multi-variate analysis through LiMMBo is more powerful in detecting genetic associations, than uni-variate tests. While the focus of this chapter was to demonstrate the applicability and power of  LiMMBo, it also highlighted the potential of multi-variate analysis for gaining insights into the underlying biology of pleiotropic loci. Analysing the effect sizes estimated by the multi-variate LMM provides the relevant data to look into common pathways and regulation and can generate hypotheses for future research.
