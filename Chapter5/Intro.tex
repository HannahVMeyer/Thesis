\chapter{GWAS of left ventricular wall thickness}
\label{section:GWAS_pheno3D}
The structure of the human heart is determined by an interplay of genetic factors and and complex environmental influences (reviewed in \citep{Payne1995, Sanoudou2005, O'Toole2008}). One common, heritable trait used to predict clinically relevant heart conditions is left ventricular mass (LVM) \citep{Post1997}. To date, genome-wide association studies (GWAS) in African American \citep{Fox2013} and Caucasian cohorts \citep{Vasan2007, Vasan2009, Arnett2009} have identified three genomic loci that are significantly associated with LVM, where LVM was assessed using echocardiographic measures or cardiac magnetic resonance (CMR) imaging. However, genetic conditions often evoke asymmetric changes in the structure of the left ventricle \citep{Chen1999, VanderMerwe2008} potentially making LVM an inaccurate phenotype for detecting genetic effects on the whole heart. To investigate genetic influences on overall heart structure instead of on a reduced representation such as LVM, spatially resolved, quantitative heart phenotypes are needed. 
A recently published method allows for the generation of such cardiac phenotypes by creating detailed 3D statistical models of the variation in the cardiac morphology \citep{DeMarvao2014}. Employing this method, De Marvao and colleagues created the first at scale cohort of 1,500 detailed 3D cardiac images from healthy volunteers.
All 3D images are mapped to a consistent volumetric reference, and over 27,000 measurements per individual representing the heart were derived. A major challenge in imaging genetics is handling the large number of correlated dimensions present in these images, even when placed in a common reference framework.  I investigated different methods to project this high dimensional phenotype space into a lower dimensional factor space as a representation of the underlying structure of the heart. These lower-dimensional projections can then be used as proxies for the heart structure in mtGWAS.