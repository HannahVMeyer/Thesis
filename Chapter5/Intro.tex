\chapter{GWAS of left ventricular wall thickness}
\label{chapter:GWAS-3Dheart}
The structure of the human heart is determined by an interplay of genetic factors and and complex environmental influences \citep{Payne1995, Sanoudou2005, OToole2008}. One common, heritable trait used to predict clinically relevant heart conditions is left ventricular mass (LVM). In particular, the increase in LVM is associated with an increased risk of heart failure and sudden death \citep{Haider1998,Post1997,Lorell2000}.  The increase in LVM through the thickening of the left ventricular wall is a direct response to a rise in hemodynamic burden which causes the hypertrophy of existing myocytes \citep{Lorell2000}. The thickening of the wall can occur in a symmetric fashion through concentric thickening of the ventricle with a small cavity dimension. However, about \num{58}\%  of all cases of left ventricular hypertrophy are asymmetric \citep{Davies1995} and the observed asymetry patterns are diverse in distribution and occurence \citep{Hughes2004,Florian2012}. A number of genetic factors have been shown to be involved in these asymmetric changes in the structure of the left ventricle \citep{Davies1995,Chen1999,VanderMerwe2008}. To date, genome-wide association studies (GWAS) in African American \citep{Fox2013}, Caucasian \citep{Vasan2007, Vasan2009, Arnett2009} and more recently Japanese cohorts \citep{Sano2016} have attempted to identify genomic loci that are significantly associated with LVM, where LVM was assessed using echocardiographic measures or 2D cardiac magnetic resonance (CMR) imaging. However, none of the studies find associations that pass the commonly applied genome-wide significance threshold. Many factors might have influenced the success of the studies and the lack of finding genetic associations such as lack in power through small sample or effect size. Given the genetic effects of the clinical LVM phenotypes observed \citep{Davies1995,Chen1999,VanderMerwe2008}, the assumptions for a genetic contribution to the natural variation in heart morphology holds, despite the negative results obtained in these studies. However, the asymmetric nature of changes in heart morphology might make LVM an inaccurate phenotype for detecting these genetic effects. To investigate genetic influences on overall heart structure instead of on a reduced representation such as LVM, spatially resolved, quantitative heart phenotypes are needed. 

A recent advance in CMR imaging is the use of 3D imaging of the heart as a whole as opposed to multiple transversal sections of the heart by 2D imaging. The latter technique has been the clinical gold standard but recent studies have shown that 3D imaging improves spatial resolution especially at the base and apex of the heart (\cref{fig:heart}) and can avoid technical issues arising from 2D imaging \citep{deMarvao2014}. Detailed images derived from the 3D imaging technique combined with genotype data would allow for an investigation into spatially-confined changes in heart morphology. Genetic association studies based on imaging phenotypes are widely applied In the field of neuroscience \citep{Filippini2009,Ho2010,Jahanshad2013,Hibar2015}. The first unbiased study using genome-wide genetic markers to find genetic associations with brain activity patterns was conducted by Stein and colleagues. They associated every voxel of 3D brain scans with all genetic markers. Following this approach, associating heart morphology as represented in the 3D scans would require testing approximately \num{140000} voxels. However, voxel-wise GWAS is limited in power and does not take into account any spatial correlation between the voxels \citep{Ge2014}. 

To overcome these limitations and offer more practical measurements for clinical use, De Marvao and colleagues have developed a technique to extract 3D features of the cardiac morphology from the 3D scans \citep{deMarvao2014}. As part of the `digital heart project' \citep{Cook2010}, they created the first at scale cohort of about \num{1500} detailed 3D statistical models of the variation in cardiac morphology from healthy volunteers. Based on these models, standard clinically relevant measurements such as LVM can be computed. Far beyond these simple 1D measurements, the 3D models allow spatially derived phenotypes such as left-ventricular wall thickness or curvature to be resolved for over \num{27000} coordinates. However, there still remains the substantial challenge in handling this still large number of correlated dimensions present in these models remain.

In the following chapter, I describe the genome-wide association study of phenotypes dervied from the 3D statistical models of the `digital heart project'. Within this project, I was responsible for the quality control and imputation of the genotypes, and conducted the GWAS from the 3D phenotypes. My colleagues collected the DNA samples, performed MRI scans and provided the 3D phenotyping. I will first describe the genotyping and phenotyping strategy and then show the results from applying different dimensionality reduction techniques to the 3D heart phenotypes. Based on the criteria described in \cref{chapter:DimReduction}, I chose the most suitable methods and conducted a GWAS with components derived thereof as proxy phenotypes. Finally, I investigated the significantly associated loci for any spatial association with the 3D heart phenotypes.

Using the genotype information which I processed and imputed, a preliminary publication on genetic associations was accepted for publication \citep{Biffi2017} and we are currently planning the publication of the analyses and results described in this chapter.


