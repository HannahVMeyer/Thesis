\newpage
\section{Successful imaging genetics of cardiac phenotypes}
In this chapter, I have described the step-wise feature extraction from high-dimensional CMR images of \num{140000} voxels to a low-dimensional  representation comprised of \num{19} components from linear and non-linear dimensionality reduction methods. The initial, atlas-based image segmentation of the original CMR images yielded reliable cardiac phenotypes at more than \num{27000} positions in the left ventricle. One such phenotype was the left ventricular wall thickness, which I transformed into a significantly lower-dimensional component space by applying a variety of dimensionality reduction methods with different properties and consequently different low-dimensional representations. Using the three measures I introduced in \cref{chapter:DimReduction}, I was able to make a prinicpled decision about which low dimensional features to retain in further investigations into the genetics. Combining all stable and trustworthy components from different dimensionality reduction methods provided a robustness to the phenotypes which allowed for qualitatively different, latent cardiac structures to be represented in the final phenotype. I successfully mapped genotypes to these \num{19} phenotypes in a multi-trait GWAS that detected three significantly associated loci. In order to link these genetic associations back to the observed wall thickness phenotypes, I associated the strongest genetic link with each wall thickness measurement and discovered a region highly associated with this SNP.  

These results are promising for genetic association studies of very high-dimensional and correlated phenotypes, as well as for this specific study on cardiac morphology. In the emerging field of imaging genetics \citep{Ge2014}, the phenotype space ranges from simple photographs of face morphology \citep{Liu2012,Shaffer2016} to  functional MRI scans of brain activity  \citep{Stein2010,Hibar2015}. While each of these phenotypes are generated by different methods and will be subjected to different challenges in acquisiton and quality control, the ultimate challenge lies in handling the high dimensionality of the phenotypes. The dimensionality reduction methods tested on the simulated data in \cref{chapter:DimReduction} and the 3D heart dataset in this chapter are all publically available and can be readily applied to any fully phenotyped dataset. 

As well as a practical example of the dimensionality reduction methods, the results of this specific combination of dimensionality reduction with GWAS are of great interest to my cardiac biology collaborators. The pre-existing cardiac related phenotypes of SKI and PRDM16 and their interaction in experimental rodent systems is very reassuring. However, before committing to further studies and publication, I will need to undertake additional manual quality control of the genotypes and ideally  would formulate additional ways to ensure the soundness of the result. Although a stringent quality control has been applied both to the actual genotypes and the imputations, poor genotype calling can lead to faulty imputations \citep{Morris2010}. I have already manually checked the genotype calling quality of \num{11377} genotypes of the Sanger12 batch, but manual quality control of the other datasets, re-imputation and potential direct genotyping of the associated SNPs should be conducted. To ensure the soundness of the result, dimensionality reduction and GWAS of 3D heart phenotypes from an independent dataset would be the ideal scenario. Unfortunately, high resolution MRI scans are not routine and even the UK BioBank MRI scans are not directly equivalent. Other possibilities include investigating the specific biology behind these loci or the specific molecular biology of the regulatory elements to provide additional evidence for the biological correctness of these associations.

