\chapter{Concluding remarks}
Initially, genome-wide association studies used seemingly simple case-control designs to map genotypes to a variety of disease phenotypes. In subsequent years, existing models were discovered for their application in GWAS \citep{Korte2012} and novel techniques developed, enabling the analysis in cohorts with complex structure \citep{Yu2006,Kang2010}, the effect estimation of sets of genotypes \citep{Wu2010,Casale2015} or gene-environment interaction in the context of GWAS \citep{Casale2017}. While sophisticated methods for the analysis of multiple traits existed \citep{Korte2012,Zhou2012,Casale2015}, they were mainly limited to moderate trait numbers due to their computational complexity. LiMMBo (\cref{chapter:limmbo}) fills this gap by enabling the joint analysis of hundreds of phenotypes.  Its performance on simulated data demonstrated its power even when only a moderate number of observed phenotypes is governed by the same genetic factors. The application to a dataset for yeast growth traits did not only show its usefulness on real data, but also demonstrated its value for investigating and generating biologically relevant hypotheses such as pleiotropy of traits and complex trait structures. 

I provide the phenotype simulation framework (\cref{chapter:simulation}) and LiMMBo as open-source software packages: \textit{PhenotypeSimulator} (\cref{chapter:simulation}) is accessible via the Comprehensive R Archive Network \citep{Meyer2017b} and LiMMBo is implemented in a python module which can be used in combination with the publicly available LIMIX suit for flexible linear mixed model designs \citep{Lippert2014}.

For very high-dimensional datasets, one is often interested in applying \textit{a priori} dimensionality reduction method to the data to extract information relevant for the biological question of interest. In the biological literature, principal component analysis is standardly employed for this task \citep{Avery2011,Liu2012,Zhang2012}. However, there exist a growing number of dimensionality reduction techniques based on different statistical methods and assumptions about the hidden data structures. Twelve of these publicly available dimensionality reduction techniques were explored for their ability to find a robust representation of the input data (\cref{chapter:DimReduction}). I used \textit{PhenotypeSimulator} to generate datasets of different sizes and underlying structures and introduced stability as a new measure to determine the dimensions of a robust low-dimensional representation. I was able to show that dimensionality reduction techniques are valuable for genotype-phenotype mapping studies of very high-dimensional datasets as the simulated genetic effects could be discovered in genetic association studies with the stable low-dimensional representations as phenotypes.

I directly applied these insights to a clincially interesting dataset of spatially-resolved three-dimensional human heart phenotypes. Based on the hypothesis that there are genetic factors that influence the heart morphology in a spatially-confined manner, I extracted low-dimensional representations of the left-ventricular wall thickness measurements and used these in a genome-wide association study. Associated SNPs did not only show a regional-confined effect but have also been implicated in cardiac phenotypes in model organisms. While further studies are needed to confirm these findings, the results demonstrate the power of this approach to investigate biologically and clinically relevant questions.

In the feature extraction approach used for this GWAS, I combined the stable low-dimensional representations from a variety of different dimensionality reduction approaches, with the underlying hypotheses that different methods capture different aspects of the morphology and a combination of the methods will yield a comprehensive representation. Alternatively, models which are more tailored to the specific structure of the dataset could be employed. The spatially-resolved heart wall thickness measurements in this study are part of a larger class of data structures, where measurements on a two-dimensional surface are embedded in a three-dimensional space. Similar data has been observed for 3D structural MRI or 4D functional MRI studies in the brain \citep{vanEssen2012,Glasser2013}. Novel feature extraction methods for neuroscience data can take \textit{a priori} knowledge about spatial correlation of the input data into account. For instance, functionalPCA combines approaches from PCA and DRR and incorporates additional sparsity priors into the model, which act on the underlying three-dimensional model of the data \citep{Lila2016}. Similar extensions could be envisaged for the Bayesian factor analysis model PEER \citep{Stegle2012}, where the spatial coordinates could be build into the model as priors.

In addition to the wall thickness measurements, the phenotyping approach developed by my collaborators also provides spatially-resolved measurement for heart wall curvature and fractional wall thickness i.e. wall thickness changes between diastole and systole. In molecular phenotyping of different tissues or conditions the simple, albeit high-dimensional genotype-phenotype mapping is extended from the two dimensional \textquote{sample by phenotype} space into the higher-dimensional \textquote{sample by phenotype by condition/tissue/etc.} space. Novel methods have been developed for the task of jointly analysing such datasets \citep{Hore2016}. These approaches could be applied to extend this study and find stable phenotype components representing a more comprehensive cardiac phenotype based on wall thickness, curvature or fractional wall thickening. 

In a second genetic association study with heart morphology, I discovered SNP-associations with a trabeculation phenotype from a supervised feature extraction approach on the raw MRI data. The implicated SNPs are located in proximity of a gene important in the developmental process of this trabeculation and follow-up studies are underway to confirm these results.

Improved diagnosis and interventional strategies in the past two decades have contributed to the general improvements in fighting cardiovascular diseases. While these improvements were mainly based on large-scale clinical trials, there is a call now for more personalised approaches to further improve the management of cardiovascular diseases \citep{Meder2016}. The proposed strategies ask for a stronger interaction between clinical, molecular and statistical expertise to enhance the characterisation of these diseases. Studies such as the GWAS on cardiac morphology show the feasibility of this proposal, with a strong collaboration between clinical and bioinformatics expertise to investigate the genetic basis of cardiac phenotypes. Follow up studies and further exploration of the data as outlined above can contribute to further characterise the genetics of cardiac structure and function. 

%(difference in thickness between end-systole and end-diastoly, normalised to the thickness at end-diastole).


