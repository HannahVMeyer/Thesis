\chapter{Conclusion}
Novel methods that can take \textit{a priori} knowledge about spatial correlation of the input data into account are currently developed. One such method is functionalPCA which combines approaches from PCA and DRR and incorporates additional sparsity priors \citep{Lila2016}. Similar extensions could be envisaged for the Bayesian factor analysis model PEER \cref{Stegle2012}, where the spatial coordinates could be build into the model as priors. 

In addition to the wall thickness measurements, the segmentation approach also provided information about wall curvature and fractional wall thickening (difference in thickness between end-systole and end-diastoly, normalised to the thickness at end-diastole). In molecular phenotyping of different tissues or conditions the simple, albeit high-dimensional genotype-phenotype mapping is extended from the \num{2} dimensional sample by phenotype space into higher-dimensional sample by phenotype by condition/tissue/etc. space. Novel methods have been developed for the task of jointly analysing these datasets \citep{Hore2016} and these approaches could be applied to extend this study and find stable phenotype components representing not only wall thinkness but in addition wall curvature or fractional wall thickening. 