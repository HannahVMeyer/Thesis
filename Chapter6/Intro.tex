\chapter{GWAS of left ventricular trabeculation}
\label{chapter:GWAS-FD}
In addition to the unsupervised phenotype selection through dimensionality reduction, the raw CMR images also provide the opportunity for a guided phenotype extraction. From a combined clinical and research point of view, phenotypes which are implicated in diseases but which also show strong natural variation are of special interest. Trabeculation phenotypes of the left ventricle fit this description.

\section{Left ventricular trabeculation}
\label{section:intro-FD}
Trabeculation is the formation of small irregular muscle protrusions from the inside of the heart wall and has its origin in early heart development. As described in \cref{section:heart-development}, the chambers of the human heart develop through the looping of the early cardiac tube. During this process, the compartimentalisation of the heart begins and the composition of the cardiac tissue changes, especially in the ventricles. At this stage, the ventricular myocardium can be described as a loose, `spongy' network of myocardial fibers that form sheet-like protrusions (trabeculae) towards the cardiac lumen. The formation of these structures supports the oxygen and nutrient exchange in the heart \citep{Chen2009} by blood flowing through the intertrabecular spaces \citep{Zambrano2002}. Later in development, the myocardium starts to become more compact and thicker and the large protrusions into the heart lumen flatten or disappear \citep{Yousef2009}. This compaction process progresses from the base of the heart towards the apex and from epicardium to endocardium \citep{Zambrano2002}

Failure of the myocardial compaction process leads to persistence of ventricular hypertrabeculation. Clincially, the majority of hypertrabeculation phenotypes are observed in the left ventricle and are refered to as left ventricular non-compaction (LVNC) \citep{Zambrano2002}. It is still unknown if LVNC constitutes a distinct disease or is a shared characteristic of different cardiomyopathies \citep{Captur2013}. Linkage studies and targeted sequencing of associated regions have revealed a number of genes implicated in familial cases of LVNC \citep{Bleyl1997,Klaassen2008,Moric-Janiszewska2008}, with a wide range of functions of the encoded proteins. These include cardiac muscle \(\alpha\) actin \citep{Monserrat2007}, 
\(\beta\)-Myosin Heavy Chain \citep{Budde2007} as well as cytoskeletal-assocoiated proteins like \(\alpha\)-dystrobrevin \citep{Ichida2001} and Cypher/ZASP \citep{Vatta2003}. Knock-out studies of genes regulating cardiovascular development have contributed to a molecular understanding of clinically relevant LVNC phenotypes \citep{Chen2009,Mysliwiec2011}. However, the genetics of sporadic LVNC remain largely unknown \citep{Zambrano2002}.

In addition to LVNC as a clinical phenotype, variation in trabeculation pattern and strength have also been observed in healthy volunteers. Several studies have analysed the range of natural and diseased non-compaction phenotypes with respect to clinical and demographic parameters \citep{Petersen2005,Captur2014}.  In particular, two independent studies have observed an increase in the ratio of non-compacted to compacted myocardium (NC:C) in individuals of African-American and Hispanic descent compared to Caucasian individuals. The lowest NC:C ratios were observed for individuals of Chinese descent \citep{Kawel2012,Captur2015}. The genetics of this natural variation and clinically observed sporadic phenotypes in humans are still poorly understood. 

In this chapter, I analyse natural genetic variation driving left ventricular trabeculation phenotypes in healthy volunteers. Trabeculation phenotypes were extracted automatically via fractal analysis from the CMR images of the healthy volunteers by my collaborators. Based on these phenotypes, I conducted a GWAS of left ventricular trabeculation.


