\chapter{GWAS of left ventricular trabeculation}
\label{chapter:FD}
Early in mammalian heart development, the myocardium is composed of a loose network of fibers and sinusoids that forms sheet-like protrusions into the cardiac lumen. These structures in the inner layer of the myocard will give rise to the trabecular myocardium. In subsequent stages of heart development, the sinusoids disappear and the trabecular myocardium becomes more compacted towards the outer wall of the myocard, forming a thick, compact ventricular wall \citep{Chen2009,Yousef2009}. Failure of the myocardial compaction process leads to persistence of ventricular hypertrabeculation or non-compaction (NC). The majority of NC phenotypes are observed in the left ventricle (LV) \citep{Zambrano2002}. It remains unknown whether LVNC is a distinct disease or a shared characteristic of different cardiomyopahties \citep{Captur2013}. In addition to NC as a clinical phenotype, variation in trabeculation pattern and strength have also been observed in healthy volunteers from different ethnic backgrounds \citep{Kawel2012,Captur2015}. In this study, we aimed to map genetic variation to LV trabeculation phenotypes. Trabeculation is quantified via fractal analysis, which meassures complex patterns. The phenotype obtained is fractal dimension (FD), a unit-mess measure, that serves as a proxy for the complexity i.e. the level of trabeculation of the endocardial wall. The higher the FD measure, the higher the complexity  of the structure. Genotype to phenotype mapping was achieved by fitting a multi-variate LMM to FD meassurements derived throughout the heart.

