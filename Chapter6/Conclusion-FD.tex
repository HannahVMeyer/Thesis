\section{Summary}
In this chapter, I used phenotypes derived from a guided feature extraction method to map naturally occurring genetic variation in healthy individuals to a clinically relevant phenotype. The association of the FD phenotypes as a quantification of left ventricular trabeculation detected two loci that are linked on a genome-wide significant level. Both loci lie in intronic regions and have no direct protein-coding consequences. Loci in proximity to the association detected within the ADAMTSL1 gene have been implicated in cardiac phenotypes such a blood pressure. However, the absence of any effect for this locus in well-powered published GWAS of blood pressure phenotypes points towards a blood pressure-independent effect on left ventricular trabeculation.

For quantitative, continuous phenotypes and additive genotype effects, understanding naturally occurring variation can give insights into the genetic architecture of the traits and might help to understand more extreme disease phenotypes. In order to extend this study and confirm results in a larger cohort, we applied for access to the UK Biobank a `large, population-based prospective study, established to allow detailed investigations of the genetic and non-genetic determinants of the diseases of middle and old age' \citep{Sudlow2015}. Within this project, \num{500000} individuals have been genotyped and phenotyped for wide array of traits, including 2D CMR scans on an expected \num{100000} individuals. In contrast to the 3D heart phenotypes investigated in \cref{chapter:GWAS-3Dheart}, the FD phenotypes can be automatically extracted from these images. Upon access to the data,  phenotype extraction and a multi-trait GWAS with the same model and parameters as described in this chapter will be conducted. 

In addition to this replication study, investigating the genetic variation driving the healthy phenotype differences in individuals of different ethnicities \citep{Kawel2012,Captur2014} will be of great interest. While the cohort in this study only contained a minority of non-Caucasian samples, a more diverse cohort structure might be observed in the UK Biobank cohort, enabling this analysis. 
