\clearpage
\chapter*{Summary}
\addcontentsline{toc}{chapter}{Summary}
\label{section:summary}

\begin{singlespace}
Over the past ten years, more than \num{4000} genome-wide association studies (GWAS) have helped to shed light on the genetic architecture of complex traits and diseases. In recent years, phenotyping of the samples has often gone beyond single traits and it has become common to record multi- to high-dimensional phenotypes for individuals. Whilst these rich datasets offer the potential to analyse complex trait structures and pleiotropic effects at a genome-wide level, novel analytic challenges arise. This thesis summarises my research into genetic associations for high-dimensional phenotype data.

First, I developed a novel and computationally efficient approach for multivariate analysis of high-dimensional phenotypes based on linear mixed models, combined with bootstrapping (LiMMBo). Both in simulation studies and on real data, I demonstrate the statistical validity of LiMMBo and that it can scale to hundreds of phenotypes. I show the gain in power of multivariate analyses for high-dimensional phenotypes compared to univariate approaches, and illustrate that LiMMBo allows for detecting pleiotropy in a large number of phenotypic traits.

Aside from their computational challenges in GWAS, the true dimensionality of very high-dimensional phenotypes is often unknown and lies hidden in high-dimensional space. Retaining maximum power for association studies of such phenotype data relies on using an appropriate phenotype representation. I systematically analysed twelve unsupervised dimensionality reduction methods based on their performance in finding a robust phenotype representation in simulated data of different structure and size. I propose a stability criteria for choosing low-dimensional phenotype representations and demonstrate that stable phenotypes can recover genetic associations.

Finally, I analysed genetic variants for associations to high-dimensional cardiac phenotypes based on MRI data from \num{1500} healthy individuals. I used an unsupervised approach to extract a low-dimensional representation of cardiac wall thickness and conducted a GWAS on this representation. In addition, I investigated genetic associations to a trabeculation phenotype generated from a supervised feature extraction approach on the cardiac MRI data.

In summary, this thesis highlights and overcomes some of the challenges in performing genetic association studies on high-dimensional phenotypes. It describes new approaches for phenotype processing, and genotype to phenotype mapping for high-dimensional datasets, as well as providing new insights in the genetic structure of cardiac morphology in humans.


\end{singlespace}


% Increasing sample sizes have allowed to extend the genotype to phenotype mapping from common to rare variants.
% far beyond standard methods that work in the regime of up to 30 phenotypes. 