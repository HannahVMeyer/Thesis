\section{LiMMBo for multi-trait GWAS and beyond}
\label{section:conclusion-limmbo}
In this chapter, I introduced \gls{limmbo}, a new method for the multivariate analysis of large trait numbers, which uses a bootstrap method to estimate complex trait covariance matrices. The main benefit of \gls{limmbo} is that it scales to \num{100}s of phenotypes, both because of its inherent sub-sampling method and that the most computationally intense part of the method can be parallelised. To take advantage of the parallelisation, I implemented an optional automatic detection for multiple cores which allows for easy realisation of this process via the \textit{Parallel Python Software} \citep{PPSoftware}. In practice, this means that trait sizes up to 30 or 40 can be in hours, rather than taking several days as for standard \gls{reml}-based methods. Most notably, complex datasets of \num{100}s of traits, which is out of scope for the \gls{reml} approaches, are feasible when using \gls{limmbo}. I showed that the covariance matrices estimated via \gls{limmbo} are as good an estimator of the real covariance matrices as the ones of the validated \gls{reml} approach. Consequently, these covariance matrices produce well calibrated null models when used in \gls{lmm} for \gls{gwas}, showing the validity of the approach. To show the advance of \gls{limmbo}, I demonstrated the power gain for multi-trait \gls{gwas} of high-dimensional phenotypes with \gls{limmbo} over standard single-trait models across a wide range of phenotype architectures. I made \gls{limmbo} accessible as an open source, python module at \url{https://github.com/HannahVMeyer/LiMMBo/tree/master/limmbo}. \gls{limmbo} is compatible with the LIMIX package for linear mixed models \citep{Lippert2014}.  

The bootstrapping has proven powerful to reduce the computational complexity for estimating the covariance parameters and made the analysis of complex datasets with high trait numbers possible. However, so far, I only examined the complexity and calibration dependent on the size of the overall phenotype set \(P\). Of additional interest would be understanding the (co-)dependence of the bootstrap size \(s\),  the number of bootstraps \(b\) and the  co-sampling of traits \(c\). Based on already simulated datasets, a systematic comparison of the run times, covariance estimates and calibration of different combinations of \(s\) and \(b\) could be conducted. For each of these combinations, different thresholds for \(c\) could be examined.

Much of the attraction of linear mixed models in genetics has been their ability to model complex genetic relatedness. As described by \citep{Kang2010} and demonstrated in this chapter, simple linear models are not suitable for analysing phenotypes with complex underlying genetic relatedness, whereas linear mixed models with the covariance matrices estimated by \gls{limmbo} are appropriate and possible up to \num{100}s of traits. Complex relatedness in populations is wide-spread in plant and animal breeding \citep{Bolormaa2014,Yang2014}, and increasingly common in human bottleneck populations \citep{Tachmazidou2013}. Furthermore, as the population numbers increase in human genetics, complex cryptic relationship structures are more prevalent \citep{Reich2001}, meaning that methods such as \gls{limmbo} will be more applicable in the future in human genetics. 

Trait-by-trait covariance matrices are useful for a variety of high dimensional big data problems across genomics, from statistical genetics to single cell analysis. The ability to accurately estimate large trait-by-trait covariance matrices using this bootstrap method may be applicable to more domains than \gls{gwas}, e.g. many gene expression studies use covariance matrices. Previous work from \citet{Schaefer2005} showed the large gene dimensions coupled with small(er) sample sets means that empirical covariance matrices could not be accurately estimated; other investigators \citep{Ledoit2004,Furrer2007,Bickel2008} used shrinkage methods to create valid covariance matrices. The work from \citet{Teng2009} uses subsampling but with strong shrinkage priors to generate the final covariance matrix. By fitting the average to closest true covariance, \gls{limmbo} ensures positive-semidefiniteness of the covariance while avoiding ill-conditioned matrices, which usually introduces large biases in the final use of these models. Thus, covariance estimation based on the method implemented in \gls{limmbo} might be applicable and useful in other areas of quantitative genetics.  

The ability to generate large cohorts of well phenotyped and genotyped individuals has forced the development of many new methods in statistical genetics. With the advent of genotyped human cohorts up to \num{500000} individuals with over \num{2000} different traits \citep{Sudlow2015}, and plant phenotyping routinely in the \num{1000}s of individuals from structured crosses with \num{100}s of (image-based) phenotypes \citep{Atwell2010,Yang2014}, new informative and scaleable methods are needed. \gls{limmbo} extends the reach of linear mixed models into this new regime, allowing for new complex genetic associations to be made.
